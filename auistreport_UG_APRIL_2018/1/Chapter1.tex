% Chapter 1

\chapter{\uppercase{Introduction}} % Main chapter title
\label{intro} % For referencing
In todays era, there are many cities which are working on transforming
themselves into Smart Cities.If the city is going to be called as Smart City,
then it should have all possible advancements in the sector of smart technology.
Improving efficiency in agriculture sector if one of the difficult and most challenging
jobs. Milk, being considered to be a staple diet among Indians, there is a lack of efficient milk delivery system. Through an android application, with one click the farmers will also be able to sell their products and the consumers will able to receive their requirements at an ease. 

Our android application can help to communicate between the farmer, distributor and customer which is implemented in this project. Android application is designed in such a way that time complexity will be minimized extensively. This is achieved by exchanging only the required data with server in order to minimize the traffic and loss of data packets in the process of delivery. With the help of cutting edge technology and keeping the goal in mind weve developed this application. It is also an attempt to
participate actively in the process of transforming into smart city and make
required services more accessible.


\section{\uppercase{Overview}}
A user can register into the application by providing their address and other details. They can update their milk requirements on a regular basis and this will get recorded in the history. They can also track their consumption, bills, check reports, change their
residential address (if they are moving to a new location) and file complaints on
distributors, if necessary.
The farmers  will also have facilities provided to the consumers, such as being able to buy milk if necessary in addition to selling, being able to track bills, transactions, etc.


